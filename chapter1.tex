\hypertarget{chapter-1---biological-sequences}{%
\section{Chapter 1 - Biological Sequences}\label{chapter-1---biological-sequences}}

\label{sec:dna}
The complete DNA sequence of an organism is called a \textit{genome}. Not every single nucleotide in a an individual's genome influences the functionality of a given organism. The functional segments of DNA are called \textit{genes}.

Although the individuals in a population of organisms of the same species share most of their DNA, there are positions at which  the sequences vary. Most of this genetic variability is caused by single nucleotide variants (SNVs), which stands for a change in a single character in some context, e.g between \texttt{ACGA} and \texttt{ACTA}. However, larger structural variants are also commonly observed, such as insertions (\texttt{AC} $\to$ \texttt{AAAAAAAAC}), deletions (\texttt{AAAAAAAAC} $\to$ \texttt{AC}), tandem repeats (adjacent copies of a substring, e.g. \texttt{ACGT} $\to$ \texttt{ACGTACGT}) and so on \cite{structVars}. A special type of SNV is a single nucleotide polymorphism (SNP), which is a change in a single nucleotide that is present with a certain frequency in a population.